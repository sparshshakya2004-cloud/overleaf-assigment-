
\subsection{Overview}
An Operating System (OS) is the most important software that runs on a computer. It manages the computer's memory and processes, as well as all of its software and hardware.

\subsection{Key Examples of Operating Systems}

There are various operating systems designed for different hardware, ranging from command-line interfaces to modern touch-based mobile systems. Key examples include:

\begin{itemize}[label=\checkmark, leftmargin=2cm]
    \item \textbf{MS-DOS:} A non-graphical command-line operating system derived from 86-DOS.
    \item \textbf{MS Windows:} Developed by Microsoft, it is the most widely used GUI-based OS for personal computers.
    \item \textbf{Android:} A Linux-based mobile OS designed primarily for touchscreen mobile devices like smartphones and tablets.
    \item \textbf{iOS:} Developed by Apple Inc. exclusively for its mobile hardware, including the iPhone and iPad.
    \item \textbf{Mac OS X (macOS):} The graphical operating system for Apple's Macintosh line of personal computers and workstations.
    \item \textbf{UNIX:} A powerful, multi-user, multitasking OS that originated in the 1970s and serves as the foundation for many modern systems.
    \item \textbf{Linux:} An open-source, Unix-like OS kernel that powers everything from servers and supercomputers to smartphones.
\end{itemize}


\newpage
\subsection{Classification by Device}
\begin{table}[h]
\centering
\renewcommand{\arraystretch}{1.8}
\begin{tabular}{|l|p{8cm}|}
\hline
\rowcolor{gray!20} \textbf{Category} & \textbf{Operating Systems} \\ \hline
Desktop / Laptop & Windows, macOS, Linux \\ \hline
Mobile / Tablet & Android, iOS \\ \hline
Server / Enterprise & UNIX, Linux, Windows Server \\ \hline
Legacy / Command-line & MS-DOS \\ \hline
\end{tabular}
\end{table}

