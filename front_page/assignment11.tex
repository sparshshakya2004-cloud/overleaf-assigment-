
\subsection{Study of the Unix Kernel}
The Unix Kernel is a classic example of a Monolithic design, though it is highly modular in nature. It is the core software that manages the hardware and provides a platform for applications.

\subsection*{Key Components of the Unix Kernel:}
\begin{itemize}
    \item \textbf{Process Management:} It handles the creation, scheduling, and termination of processes. It uses a "Scheduler" to decide which process gets the CPU and for how long.
    \item \textbf{Memory Management:} It allocates memory to processes and ensures that one process doesn't overwrite the memory of another (Memory Protection).
    \item \textbf{File System Management:} Unix treats everything as a file (including hardware devices). The kernel manages the data structure on the disk and handles read/write requests.
    \item \textbf{System Calls:} This is the interface between the user programs and the kernel. If an app wants to open a file, it must send a "System Call" to the kernel to get permission.
\end{itemize}

\subsection*{The "Two-Part" Philosophy:}
The Unix system is traditionally divided into two parts:
\begin{itemize}
    \item \textbf{The Kernel:} Which handles the hardware and low-level tasks.
    \item \textbf{The Shell:} The environment that sits around the kernel, taking user commands and passing them to the kernel for execution.
\end{itemize}

\newpage

\subsection{Comparison of Various Types of Kernels}
Different kernels are designed based on whether the priority is speed, size, or reliability.



\renewcommand{\arraystretch}{2}
\begin{longtable}{|p{2.5cm}|p{4cm}|p{3.5cm}|p{3cm}|}
\hline
\rowcolor{gray!20} \textbf{Kernel Type} & \textbf{Description} & \textbf{Key Characteristic} & \textbf{Example} \\ \hline
\endfirsthead

\textbf{Monolithic Kernel} & The entire OS runs in the kernel space. All drivers, memory management, and file systems are one big block. & Fastest performance because everything is in one place. & Linux, BSD, Unix. \\ \hline

\textbf{Microkernel} & Only the bare essentials (IPC, basic memory) are in the kernel. Everything else (drivers, etc.) runs in "User Space." & High Stability; if a driver crashes, the system stays up. & QNX, L4, Mach. \\ \hline

\textbf{Hybrid Kernel} & A mix of both. It looks like a microkernel but runs some code as monolithic to improve speed. & Balance of speed and modularity. & Windows NT, macOS (XNU). \\ \hline

\textbf{Exokernel} & A minimalist kernel that gives applications direct access to hardware resources. & Ultra-flexible for developers to optimize their own apps. & MIT Exokernel. \\ \hline

\textbf{Nanokernel} & A tiny kernel that only supports hardware interrupts and nothing else. & Extremely small; usually for specialized embedded systems. & KeyKOS. \\ \hline

\end{longtable}

% \end{document}